\documentclass[a4paper, 9pt]{extarticle}

\usepackage{titling}
\usepackage{titlesec}
\usepackage[margin=15mm]{geometry}

\usepackage[turkish]{babel}
\usepackage[T1]{fontenc}
\usepackage[utf8]{inputenc}

\usepackage{anyfontsize}
% \usepackage{noto}
\usepackage{times}

\usepackage{multicol}
\usepackage{parskip}
\usepackage[hidelinks]{hyperref}

\usepackage{xpatch}

%% Title Formatting %%%%%%%%%%%%%%%%%
\titleformat{\section}
{\large}
{}
{0em}
{\bfseries}[\titlerule]

\titleformat{\subsection}
{\normalsize}
{}
{0em}
{\bfseries\underline}

\titleformat{\subsubsection}[runin]
{\small}
{\hspace{0em}}
{0em}
{\bfseries}[]

%\titlespacing{\subsubsection}
%{0em}
%{0em}
%{0em}
%%%%%%%%%%%%%%%%%%%%%%%%%%%%%%%%%%%%%

\renewcommand{\maketitle}{
    {\raggedright\large\bfseries\myname}
    {\hfill \mygmail}

    {\hfill\myphone}
}

\newcommand{\makebaslik}{
    {\raggedright\large\bfseries\myisim}
    {\hfill \mygmail}

    {\hfill\myphone}
}

%% Definitions %%%%%%%%%%%%%%%%%%%%
\def\myphone{0538 041 40 30}
\def\mygmail{yigitemres@gmail.com}
\def\mypm{yigitemres@pm.me}
\def\mygradyear{2011 --- 2016}
\def\mythm{https://tryhackme.com/p/yigitemres}
%% Turkish %%%%%%%%%%%%%%%%%%%%%%%%
\def\myisim{Yiğit Emre Şahinoğlu}
\def\myadres{
    Soğuksu Mahallesi Karadut Sokak
    Kardelen Konutları 6/1 D.6 Beykoz/İstanbul
    }
\def\myuniversite{Bahçeşehir Üniversitesi}
\def\myfakulte{Mühendislik ve Doğa Bilimleri Fakültesi}
\def\mydepartman{Bilgisayar Mühendisliği (İngilizce)}

%% English %%%%%%%%%%%%%%%%%%%%%%%%%
\def\myname{Yigit Emre Sahinoglu}
\def\myaddress{
    Soguksu Mahallesi Karadut Sokak
    Kardelen Konutlari 6/1 D.6 Beykoz/Istanbul
    }
\def\myuniversity{Bahcesehir University}
\def\myfaculty{Faculty of Engineering and Natural Sciences}
\def\mydepartment{Computer Engineering (English)}
%%%%%%%%%%%%%%%%%%%%%%%%%%%%%%%%%%%%%%%%%%%%%%

\def\myeducation{
    \section{Education}
    {\bfseries \myuniversity{} \hfill \myfaculty{}}\\
    {\mydepartment{} \hfill \mygradyear{}}
}

\def\myegitim{
    \section{Eğitim}
    {\bfseries \myuniversite{} \hfill \myfakulte{}}\\
    {\mydepartman{} \hfill \mygradyear{}}
}

\def\myexperience{
    \section{Experience}
    {\bfseries Besiktas Municipality}\\
    Internship \hfill 2016
}

\def\mydeneyim{
    \section{Deneyim}
    {\bfseries Beşiktaş Belediyesi}\\
    Staj \hfill 2016
}

\def\mylanguages{
    \section{Languages}
	Turkish (L1), English (B2+), Russian (A2)
	}

\def\mydiller{
    \section{Diller}
	Türkçe (Anadil --- L1), İngilizce (Akademik --- B2+), Rusça (Başlangıç --- A2)
}

\def\thmrank{
    TryHackMe Rank: 1573 (37308)
}

\def\thmderece{
    TryHackMe Derecesi: 1573 (37308)
}

\def\cybersec{
	Network Exploitation\\
	Privilege Escalation\\
	Web Exploitation\\
	\thmrank{}
}

\def\siberguvenlik{
	Network Exploitation\\
	Privilege Escalation\\
	Web Exploitation\\
	\thmderece{}
}

\def\mysertifikalar{
	\section{Sertifikalar}
	\begin{minipage}[t]{0.70\textwidth}
		Udemy --- Broad Scope Bug Bounties From Scratch\\
		Udemy --- Practical Ethical Hacking - The Complete Course\\
		Udemy --- The Ultimate BAC and IDOR Guide for Ethical Hacking\\
		Udemy --- XSS Survival Guide\\
		Wissen Akademi --- Cloud Computing and Its Applications\\
		Wissen Akademi --- Network Systems Engineering\\
		Wissen Akademi --- Advanced Network Systems Engineering\\
    \end{minipage}
}

\def\mycertificates{
	\section{Certificates}
	\begin{minipage}[t]{0.70\textwidth}
		Udemy --- Broad Scope Bug Bounties From Scratch\\
		Udemy --- Practical Ethical Hacking - The Complete Course\\
		Udemy --- The Ultimate BAC and IDOR Guide for Ethical Hacking\\
		Udemy --- XSS Survival Guide\\
		Wissen Akademi --- Cloud Computing and Its Applications\\
		Wissen Akademi --- Network Systems Engineering\\
		Wissen Akademi --- Advanced Network Systems Engineering\\
    \end{minipage}
}

\def\myhobbies{
    \section{Hobbies and Interests}
	Card Magician\\
	Fishing\\
	Leathercraft\\
	Pen Enthusiast\\
	Playing Card Collection\\
	Watch Enthusiast\\
	Wet Shaving Enthusiast
}

\def\myhobiler{
    \section{Hobiler ve İlgi Alanları}
	Amatör Balıkçılık\\
	Amatör İllüzyon\\
	Geleneksel Tıraş\\
	İskambil Kağıdı Koleksiyonu\\
	Kalem Koleksiyonu\\
	Saraciye
}

\def\mytechnicalskillsb{
    \section{Technical Skills}
    \vspace{-0.19em}
    \begin{minipage}[t]{0.45\textwidth}
		\subsubsection{Currently}
        Bash/Powershell\\
        Emacs\\
        ELisp/CLisp (Emacs Lisp/Common Lisp)\\
        GNU/Linux (Gentoo \& Arch)\\
        Portage (Gentoo Package Manager)\\
        Git\\
        \LaTeX\\
        Make\\
        Python 3\\
        Reverse Engineering (Beginner)\\
    \end{minipage}
	\begin{minipage}[t]{0.45\textwidth}
		\subsubsection{Cyber Security}
        \cybersec{}\\
    \end{minipage}
    \begin{minipage}[t]{0.45\textwidth}
		\subsubsection{Formerly}
        C/C++\\
        Go\\
        Java\\
        Kotlin\\
        Network\\
        Vim\\
        Visual Studio\\
    \end{minipage}
}

\def\myteknikbecerilerb{
    \section{Teknik Beceriler}
    \vspace{-0.19em}
    \begin{minipage}[t]{0.45\textwidth}
		\subsubsection{Güncel}
        Bash/Powershell\\
        Emacs\\
        ELisp/CLisp (Emacs Lisp/Common Lisp)\\
        GNU/Linux (Gentoo \& Arch)\\
        Portage (Gentoo Package Manager)\\
        Git\\
        \LaTeX\\
        Make\\
        Python 3\\
        Reverse Engineering (Beginner)\\
    \end{minipage}
	\begin{minipage}[t]{0.45\textwidth}
		\subsubsection{Siber Güvenlik}
        \siberguvenlik{}\\
    \end{minipage}
    \begin{minipage}[t]{0.45\textwidth}
		\subsubsection{Daha Önce}
        C/C++\\
        Go\\
        Java\\
        Kotlin\\
        Network\\
        Vim\\
        Visual Studio\\
    \end{minipage}
}

\def\myhobbiesb{
    \section{Hobbies and Interests}
    \vspace{-0.19em}
    \begin{minipage}[t]{0.45\textwidth}
        Card Magician\\
        Fishing\\
        Leathercraft\\
        Pen Enthusiast\\
    \end{minipage}
    \begin{minipage}[t]{0.45\textwidth}
        Playing Card Collection\\
        Watch Enthusiast\\
        Wet Shaving Enthusiast
    \end{minipage}
}

\def\myhobilerb{
    \section{Hobiler ve İlgi Alanları}
    \vspace{-0.19em}
    \begin{minipage}[t]{0.45\textwidth}
        Amatör Balıkçılık\\
        Amatör İllüzyon\\
        Geleneksel Tıraş\\
        Kalem Koleksiyonu\\
    \end{minipage}
    \begin{minipage}[t]{0.45\textwidth}
        Saat Koleksiyonu\\
        Saraciye (Dericilik)\\
        İskambil Kağıdı Koleksiyonu
    \end{minipage}
}


\begin{document}

\makebaslik{}
\myegitim{}
\mydeneyim{}

\section{Bitirme Projesi}

\subsection{Güvenliği Artırılmış Mesajlaşma Platformu (Python 3)}
\hfill \\
\begin{itemize}
	\item Diffie-Hellman kütüphanesi kullanıldı.
	\item Sistem yormaması açısından MongoDB tercih edildi.
	\item WebSocket kullanılarak server oluşturuldu.
\end{itemize}

\section{Kişisel Projeler}

\subsection{IBBApi (C++)} \hfill \\
\begin{itemize}
	\item Mevcut İBB API'larına cli üzerinden erişir.
	\item Proje ilk başladığında balık ve sebze-meyve halinden fiyat çekip,
	\underline{\emph{\hyperref[lisp:frp]{Yemek Tarif Paketi (ELisp)}}}'nden mevcut malzeme giderlerinin hesaplanması için yapıldı.
\end{itemize}

\subsection{4 Yönlü Ballistik Yazılımı (C++)} \hfill \\
\begin{itemize}
	\item 4 Yönlü serbestlik derecesinde (DOF) ballistik hesaplama yapar.
	\item Rüzgar sapma değerleri de mevcuttur.
\end{itemize}

\subsection{Yemek Tarifi Paketi (ELisp)} \label{lisp:frp} \hfill \\
\begin{itemize}
	\item Emacs içinden yemek tariflerinin gösterilmesini ve ilgili tarifin
	malzemelerine dinamik değerler atanmasını sağlar.
	\item Bu paketin amacı yemek tariflerinde ana malzemeye göre belirlenen
	oranlarda (bknz: Ekmek yapımında malzemeler un ağırlığına göre değişkenlik
	gösterir) olan değişimin, değişkenlerle hesaplamasını yapıp kullanıcıya
	göstermektir. Malzemelerin gramajları statik değil dinamik olarak belirlenir.
\end{itemize}

\subsection{Metrik $\longleftrightarrow$ Emperyal Birim Çevirici (ELisp)}
\hfill \\
\begin{itemize}
	\item Emperyal birimleri metrik birimlere çevirmeyi sağlar.
	\item Ayrıca Emacs'te seçili bölgede ilgili emperyal birimleri tespit edip
	otomatik olarak birim çevirimi yapar.
	% Eğer bulunan emperyal birimlerin metrik birimlere çevriminde birden fazla
	% birim seçeneği varsa (fit -> cm yada mm) önceden belirlenmiş kurala göre en
	% uygun olanına çevirim yapar.
\end{itemize}

% \subsection{Youtube-DLP Paketi (ELisp)} \hfill \\
% \begin{itemize}
	% \item Emacs içinden \emph{yt-dlp} python paketine komut gönderimi sağlar.
	% \end{itemize}

\subsection{Yee Light Paketi (ELisp)} \hfill \\
\begin{itemize}
	\item Emacs içinden \underline{\emph{\href{https://gitlab.com/stavros/yeecli/-/tree/master}{yeecli}}} python paketine komut gönderimi sağlar.
\end{itemize}

\subsection{Haber Toplayıcı (News Aggregator) (Python 3 \& Go)} \hfill \\
\begin{itemize}
	\item Inoreader'ın yerine kişisel kullanım için yapıldı.
	\item Go versiyonu Python projesine alternatif olarak Go öğrenmek için yapıldı.
	\item Sadece PC'de kullanılabildiği için sonlandırıldı.
\end{itemize}

\subsection{Kişisel Dosya Yedekleyici (Python 3)} \hfill \\
\begin{itemize}
	\item \emph{Dotfiles} yedeklemesini kolaşlaştırmak için yapıldı.
	\item 2 yıl kullanıldı.
	\item Yerini "\emph{git init --bare root}" yöntemi aldı.
\end{itemize}

\newpage

\mydiller{}
\myteknikbecerilerb{}
\mysertifikalar{}
% \myhobilerb{}

\pagenumbering{gobble}
\end{document}
